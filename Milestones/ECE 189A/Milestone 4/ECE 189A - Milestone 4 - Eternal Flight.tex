% Preamble
\documentclass[paper=a4, fontsize=10pt]{scrartcl}
% Settings for the document
\setlength{\paperheight}{11in}
\setlength{\paperwidth}{8.5in}

% Packages to include
\usepackage{amsmath}
\usepackage{color}
\usepackage[left=0.75in, right=0.75in, top=0.75in, bottom=0.75in]{geometry}
\usepackage{graphicx}
\usepackage{mathptmx}
\usepackage{mathtools}

% Location for figures
\graphicspath{{./figures/}}

%%%%% MY COMMANDS %%%%%


% Maketitle Metadata
\title{
		\vspace{-1in} 	
		\usefont{OT1}{bch}{b}{n}
		\normalfont \large \textsc{ECE 189A - Eternal Flight} \\ [10pt]
		\rule{\linewidth}{2pt} \\ [0.4cm]
		\Huge Milestone 4 \\
		\LARGE Detailed Design Checklist \\
		\rule{\linewidth}{2pt}
}
\author{
		\normalfont 
		\large Richard Boone, Kyle Douglas, Sayali Kakade, \\
		\large Sang Min Oh, \& Aditya Wadaskar \\ [-3pt]		 			%\large \today
}
\date{}

% Begin document
\begin{document}
\maketitle

%%%%%%%%%%%% BEGIN DOCUMENT CONTENT %%%%%%%%%%%%
\vspace{-0.5in}%
\section{Intended Software Structure}
\subsection*{Overall Structure}
The software structure of each part of the project is based on what we plan to demonstrate at the Design Expo in May 2019. The following list describes our plan for the final demonstration:
\begin{enumerate}
	\item Both the parent and child drones are at the base station.
	\item The child drone is flown to point A using radio control. The child remains stationary and hovers at point A while the parent drone is still at the base station.
	\item The parent drone is activated -- it is now able to communicate with the child drone over WIFI.
	\item The child communicates its GPS coordinates (point A) to the parent.
	\item The parent drone flies to $N$ feet directly below the child drone at point A.
	\item The child drone detects the parent drone, descends, and lands on the surface of the parent drone. 
	\item The parent drone replaces the child drone's stale battery.
	\item Once the battery-switching is complete, the child drone unlatches from the surface of the parent drone and flies back to point A.
	\item The parent drone is flown back to the base station using radio control, where we replace both its battery and a full battery for the child drone.
	\item We repeat steps 3-9. The entire time, the child drone is hovering at point A.
	\item If we want to land the child drone, we can take control of it and bring it back to the base station using radio control. If desired, we can restart the demonstration from step 1. Furthermore, there will be battery indicators for both the parent and child drones in case of emergencies.
\end{enumerate}

\newpage
\subsection*{Parent Drone}
The parent drone has one on-board computer, a Raspberry Pi 3 B+, which will run Raspbian Lite (a headless Debian-based Linux distribution). The following table outlines the hardware-software interactions and the layout of the software with respect to each of the modules that will need to be programmed for interaction with the Raspberry Pi.\\%
\begin{table}[h!]
	\centering
	\begin{tabular}{|l|p{0.6\textwidth}|}
		\hline
		\textbf{Hardware Connections to Software} & \textbf{Purpose of Software}\\%
		\hline
		\hline
		USB1 $\rightarrow$ DJI N3 Flight Controller & UART connection: sends directional instructions to the parent drone to control its movement. We will explore the exact interfacing and API calls further.\\%
		\hline
		USB2 $\rightarrow$ Wifi Adapter & USB connection: acts as a fail-safe for the DGPS module. It allows remote access to the Raspberry Pi for other communications between the parent and child drones, if necessary.\\%
		\hline
		USB3 $\rightarrow$ ublox DGPS Module & USB connection: provides RTK positioning, with the parent drone acting as the moving baseline in relation to the child drone.\\%
		\hline
		GPIO Pin 2 $\rightarrow$ Linear Actuator & Applies a linear force to move batteries in and out of the battery-switching contraption on the child drone.\\%
		\hline
	\end{tabular}
\end{table}
\hfill\\%
The structure of the software on the Raspberry Pi 3 B+ will be as follows:\\%
\\%
\textbf{State 1: Deactivated -- The parent drone is not searching for the child drone}\\%
\begin{algorithm}[H]
	\While{parent drone not activated}{
		ignore child drone\;
	}
	transition to state 2\;
\end{algorithm}
\hfill\\%
\textbf{State 2: Activated -- The parent drone is actively searching for the child drone}\\%
\begin{algorithm}[H]
	\While{no communication with child drone}{
		establish communication over WIFI with child drone\;
	}
	retrieve GPS coordinate of child drone\;
	fly to \textit{N} feet below child drone and hover\;
    activate electromagnets in preparation for the child drone landing\;
	\While{child drone not latched to parent}{
		hover in place\;
	}
	transition to state 3\;
\end{algorithm}
\hfill\\%
\textbf{State 3: Battery Switching -- The child drone is latched onto the parent drone}\\%
\begin{algorithm}[H]
	activate linear actuator\;
	insert new battery into child drone and push out stale battery\;
	signal child drone to power on and unlatch from parent drone\;
	\While{signal acknowledgment not received from child drone}{
		keep electromagnets activated\;
	}
	deactivate electromagnets\;
	\While{child is latched onto parent}{
		hover in place\;
	}
	transition back to state 1\;
\end{algorithm}

\newpage
\subsection*{Child Drone}
The child drone will also have an on-board computer, a Raspberry Pi Zero W, which will also run Raspbian Lite. The child drone will have a PixRacer flight controller, which will be running the px4 flight control framework. The following table outlines the hardware-software interactions and the layout of the software for the child drone's on-board computer.
\begin{table}[h!]
	\centering
	\begin{tabular}{|p{0.35\textwidth}|p{0.6\textwidth}|}
		\hline
		\textbf{Hardware Connections to Software} & \textbf{Purpose of Software}\\%
		\hline
		\hline
		USB $\rightarrow$ Telemetry1 of PixRacer & Serial port connection: controls the movement of the child drone motors using directional commands.\\%
		\hline
		GPIO 4, 6 $\rightarrow$ Step-Down Converter & Steps-down the battery voltage from 14.7 V to 5 V for powering other peripherals.\\%
		\hline
		GPIO 8, 10 $\rightarrow$ OpenMV P4/P5 (TX/RX) & $\text{I}^2\text{C}$ connection: communicates the location in which the AprilTag (and thus, the parent drone) is detected.\\%
		\hline
		UART $\rightarrow$ ublox DGPS Module & Gets the current location of the parent drone with the child drone acting as the rover in the ublox RTK moving baseline model.\\%
		\hline
	\end{tabular}
\end{table}
\hfill\\%
The structure of the software on the Raspberry Pi Zero W will be as follows:
\hfill\\%
\textbf{State 1: Deactivated -- The parent drone is not searching for the child drone}\\%
\begin{algorithm}[H]
	\While{parent drone not activated}{
		remain stationary\;
	}
	transition to state 2\;
\end{algorithm}
\hfill\\%
\textbf{State 2: Activated -- The child drone is hovering and waiting for the parent drone}\\%
\begin{algorithm}[H]
	\While{AprilTag not detected}{
		hover in place\;
	}
	\While{child drone not latched to parent drone}{
		read AprilTag information from OpenMV\;
		provide directions to PixRacer to descend and land on parent drone\;
	}
	signal to the parent drone indicating successful latch\;
	transition to state 3\;
\end{algorithm}
\hfill\\%
\textbf{State 3: Battery Switching -- The child drone is latched onto the parent drone}\\%
\begin{algorithm}[H]
	\While{signal to unlatch not received from parent drone}{
		remain stationary on top of the parent drone\;
	}
	supply power to motors\;
	send acknowledgment of signal to parent drone\;
	rapidly take off from the parent drone\;
	transition back to state 2\;
\end{algorithm}	

\newpage
\section{Schematics}
\begin{figure}[h!]
	\centering
	\includegraphics[width=0.87\textwidth]{parentedit}
	\caption{Parent Drone Schematic}
\end{figure}
\newpage
\begin{figure}[h!]
	\centering
	\includegraphics[width=0.95\textwidth]{childedit}
	\caption{Child Drone Schematic}
\end{figure}

%%%%%%%%%%%%% END DOCUMENT CONTENT %%%%%%%%%%%%%

%%%%%%%%%%%%%%% BEGIN REFERENCES %%%%%%%%%%%%%%%

%%%%%%%%%%%%%%%% END REFERENCES %%%%%%%%%%%%%%%%

% End document
\end{document}